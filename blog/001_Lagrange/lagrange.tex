\documentclass[a4paper]{article}
\setlength{\parindent}{0em}
\setlength{\parskip}{1em}

\usepackage[utf8]{inputenc}

\usepackage{amsmath}
\usepackage{amsfonts}
\usepackage{units}

\usepackage{graphicx}
\graphicspath{ {images/} }


\title{\includegraphics[scale=0.3]{aragon.pdf}\\
  \; \\
  Lagrange bases in subgroups of $\mathbb{F}_p^{*}$ :\\
  a hands-on introduction \\
  \small \; \\
  \small Aragon Research - Math Seminar Note \#1}
\author{Alex Kampa}
\date{April 2022}

\begin{document}

\maketitle

\textit{This seminar note aims to provide an easy to follow introduction to Lagrange
bases in the particular context of subgroups of $\mathbb{F}_p^{*}$. Readers are encouraged to redo some of the examples by hand.}


% ================================================================================

\section{Setting and Motivation}

We are in the field $\mathbb{F}_p$ where $p$ is prime. The element $\omega \neq 0$ is a generator of order $n$ of a multiplicative subgroup $H$ of $\mathbb{F}_p^{*}$. Obviously, $n$ divides $p-1$, which is the order of $\mathbb{F}_p^{*}$, and we have:

\begin{equation}
    H = \{ 1, \omega, \omega^2, ..., \omega^{n-1} \}
\end{equation}

We seek to represent a polynomial $P(x)$ over $H$ that takes a set of predefined values $v_i$ over the elements of $H$. In other words, given a set $V = \{v_0, v_1, ..., v_{n-1}\}$, we seek a polynomial $P$ such that:
\begin{equation}
    \forall \: \omega^i \in H, \; P(\omega^i) = v_i
\end{equation}

Lagrange polynomials $L_i$ provide an easy way to do this. The desired $P$ is simply expressed as:

\begin{equation}
    P(x) = v_0 L_0(x) + v_1 L_1(x) + ... + v_{n-1} L_{n-1}(x)
\end{equation}

Lagrange polynomials, also called Lagrange bases, provide an alternative and useful approach to polynomials: instead of defining them by their coefficients, they are defined by their values. Lagrange polynomials have recently been used in the construction of a popular zk-SNARK scheme called PLONK \cite{DBLP:journals/iacr/GabizonWC19}.


% ================================================================================

\section{Definitions}

The Lagrange polynomials on $H$ are a set of polynomials $L_i$ defined for $0 \le i < n-1$ as follows:

\begin{equation}
    \forall x \in H, \;\; L_i(x) =
    \begin{cases}
      1 \text{   for } x = \omega^i\\
      0 \text{   otherwise}
    \end{cases} 
\end{equation}

It should be clear that we can write $L_i(x)$ as:

\begin{equation}
    L_i(x) = \alpha_i \prod_{\substack{j=0 \\ j \neq i}}^{n-1} (x - \omega^j)
\end{equation}

It is also useful to define the polynomial $L(x)$ which has roots at exactly all the element of H:

\begin{equation}
    L(x) = \prod_{j=0}^{n-1} (x - \omega^j)
\end{equation}

\textbf{Our principal aim is to show that:}

\begin{equation}
    L(x) = x^n - 1
\end{equation}

\textbf{and that:}

\begin{equation}
    L_i(x) = \frac{\omega^i}{n} \cdot \frac{x^n - 1}{x - \omega^i} = \frac{1}{n} \cdot \sum_{k=0}^{n-1} \omega^{-ik} x^k
\end{equation}


% --------------------------------------------------------------------------------

\subsection{Example 1}

We place ourselves in $\mathbb{F}_3^{*}$, with $\omega = 2$, $H = \{1, 2\}$ and therefore $n = 2$. The two Lagrange polynomials in this case can be written:

\begin{equation}
  \begin{aligned}
    L_0(x) &= a_0(x - 2) \\
    L_1(x) &= a_1(x - 1) \\
  \end{aligned}
\end{equation}

As a result, and remembering that we are computing modulo $3$, we find that:

\begin{equation}
  \begin{aligned}
    L_0(1) &= a_0(1 - 2) = 2a_0 = 1 \implies a_0 = 2 = \nicefrac{1}{2} \\
    L_1(2) &= a_1(2 - 1) = 1a_1 = 1 \implies a_1 = 1 = \nicefrac{2}{2} \\
  \end{aligned}
\end{equation}

Now let's compute $L(x)$, noting that we have $2 = -1$ in modulo 3:

\begin{equation}
  \begin{aligned}
    L(x) &= (x-1)(x-2) \\
         &= x^2 - x(1+2) + 1\cdot2 \\
         &= x^2 - 3x + 2 \\
         &= x^2 - 1
  \end{aligned}
\end{equation}

% --------------------------------------------------------------------------------

\subsection{Example 2}

We place ourselves in $\mathbb{F}_7^{*}$, with $\omega = 2$, $H = \{1, 2, 4\}$ and therefore $n = 3$. The three Lagrange polynomials in this case can be written:

\begin{equation}
  \begin{aligned}
    L_0(x) &= a_0(x - 2)(x-4) \\
    L_1(x) &= a_1(x - 1)(x-4)\\
    L_2(x) &= a_2(x - 1)(x-2)\\
  \end{aligned}
\end{equation}

As a result, and remembering that we are computing modulo $7$, we find:

\begin{equation}
  \begin{aligned}
    L_0(1) &= a_0(1 - 2)(1 - 4) = 3a_0 = 1 \implies a_0 = 5 = \nicefrac{1}{3} \\
    L_1(2) &= a_1(2 - 1)(2 - 4) = 5a_1 = 1 \implies a_0 = 3 = \nicefrac{2}{3} \\
    L_2(4) &= a_2(4 - 1)(4 - 2) = 6a_2 = 1 \implies a_0 = 6 = \nicefrac{4}{3} \\
  \end{aligned}
\end{equation}

Now let's compute $L(x)$:

\begin{equation}
  \begin{aligned}
    L(x) &= (x-1)(x-2)(x-4) \\
         &= x^3 - x^2(1+2+4) + x(1\cdot2 + 1\cdot4 + 2\cdot4) - 1\cdot2\cdot4 \\
         &= x^3 - 7x^2 + 14x - 8 \\
         &= x^3 - 1
  \end{aligned}
\end{equation}


% ================================================================================

\section{More Definitions}

We assume that $p$ and $\omega$ are fixed, and therefore also $n$. We define $I$ to be the set containing all $n$ integers between $0$ and $n-1$.

\begin{equation}
  I = \{0, 1, ..., n-1\}
\end{equation}

For $i \in I$, we define $I_i$ as the set of all integers between $0$ and $n-1$, with the exception of $i$.

\begin{equation}
  I_i = \{0, 1, ..., n-1\} \; \backslash \; \{i\}
\end{equation}

For $0 < k \le n-1$, We define $C(k)$ as the set of all strictly increasing sequences of length $k$ contained in $I^k$. 

\begin{equation}
  C(k) = \{ \{j_1, ..., j_k\} \in I^k : j_1 < .. < j_k \}
\end{equation}

Clearly, $C(k)$ represents the number of $k$-element subsets of $I$, and $|C(k)| = \binom{n}{k}$. We similarly  define $C_i(k)$ as the set of all strictly increasing sequences of length $k$ contained in $I_i^k$.

\begin{equation}
  C_i(k) = \{ \{j_1, ..., j_k\} \in I_i^k : j_1 < .. < j_k \}
\end{equation}

We can now define the following sums over elements of $C(k)$ and $C_i(k)$, with $k > 0$:

\begin{equation}
  \begin{aligned}
    S(k) = \sum_{\vec{j} \in C(k)} \omega^{j_1 + .. + j_k}
  \end{aligned}
\end{equation}

\begin{equation}
  \begin{aligned}
    S_i(k) = \sum_{\vec{j} \in C_i(k)} \omega^{j_1 + .. + j_k}
  \end{aligned}
\end{equation}

Finally, it is useful to define these sums for $k=0$ as follows:

\begin{equation}
    S(0) = S_i(0) = 1
\end{equation}

Note that in all the above definitions, $n$ is considered as fixed. If necessary the sets/sums defined above could be denoted more explicitly as $I(n)$, $I_i(n)$, $C(k,n)$, $C_i(k,n)$, $S(k,n)$ and $S_i(k,n)$.


% --------------------------------------------------------------------------------

\subsection{Example 3}

Let's assume our subgroup is of order 5. This would be the case if we took $\omega = 3$ in $\mathbb{F}_{11}^{*}$, with $\{ 1, \omega, \omega^2, \omega^3, \omega^4 \} = \{1, 3, 9, 5, 4\}$. The polynomial $L(x)$ can be written as:

\begin{equation}
  \begin{aligned}
    L(x) &= (x - \omega^0)(x - \omega^1)(x - \omega^2)(x - \omega^3)(x - \omega^4) \\
         &= a_5 x^5 + a_4 x^4 + a_3 x^3 + a_2 x^2 + a_1 x + a_0
  \end{aligned}
\end{equation}

The coefficients of this polynomial are:

\begin{equation}
  \begin{aligned}
    a_5 &= 1 \\
        &= (-1)^{0} \cdot S(0) \\
    \\
    a_4 &= -(\omega^0 + \omega^1 + \omega^2 + \omega^3 + \omega^4) \\
        &= (-1)^{1} \cdot S(1) \\
    \\
    a_3 &= \omega^0\omega^1 + \omega^0\omega^2 + \omega^0\omega^3 + \omega^0\omega^4 + \omega^1\omega^2 \\
        &  + \omega^1\omega^3 + \omega^1\omega^4 + \omega^2\omega^3 + \omega^2\omega^4 + \omega^3\omega^4 \\
        &= (-1)^{2} \cdot S(2) = 2 \cdot (\omega^0 + \omega^1 + \omega^2 + \omega^3 + \omega^4)\\
    \\
    a_2 &= -(\omega^0\omega^1\omega^2 + \omega^0\omega^1\omega^3 + \omega^0\omega^1\omega^4 + \omega^0\omega^2\omega^3 +\omega^0\omega^2\omega^4\\
        &+ \omega^0\omega^3\omega^4 + \omega^1\omega^2\omega^3 + \omega^1\omega^2\omega^4 +\omega^1\omega^3\omega^4 +\omega^2\omega^3\omega^4) \\
        &= (-1)^{3} \cdot S(3) = -2 \cdot (\omega^0 + \omega^1 + \omega^2 + \omega^3 + \omega^4)\\
    \\
    a_1 &= \omega^0\omega^1\omega^2\omega^3 + \omega^0\omega^1\omega^2\omega^4 + + \omega^0\omega^1\omega^3\omega^4 \\
        &+ \omega^0\omega^2\omega^3\omega^4 + \omega^1\omega^2\omega^3\omega^4 \\
        &= (-1)^{4} \cdot S(4) = \omega^0 + \omega^1 + \omega^2 + \omega^3 + \omega^4 \\
    \\
    a_0 &= - \omega^0\omega^1\omega^2\omega^3\omega^4 \\
        &= (-1)^{5} \cdot S(5) \\
  \end{aligned}
\end{equation}

Note that $a_4$ is a sum of $5 = \binom{5}{1}$ elements, $a_3$ a sum of $10 = \binom{5}{2}$ elements etc.

Also note that, because $\omega^5 = \omega^0 = 1$, we have:

\begin{equation}
  \begin{aligned}
   \omega \cdot a_4 &= \omega \cdot (\omega^0 + \omega^1 + \omega^2 + \omega^3 + \omega^4) \\
                    &= \omega^1 + \omega^2 + \omega^3 + \omega^4 + \omega^0 \\
                    &= a_4
  \end{aligned}
\end{equation}

Now $\omega \cdot a_4 = a_4$ implies $a_4 = 0$ because $\omega \neq 0$. As a result, it is clear that $a_1 = a_2 = a_3 = a_4 = 0$. As to $a_0$, we have:

\begin{equation}
  a_0 = -\omega^{0+1+2+3+4} = -\omega^{10}  = -1
\end{equation}

This conforms to our expectation that:

\begin{equation}
  L(x) = x^5 - 1
\end{equation}


% --------------------------------------------------------------------------------

\subsection{Example 4}

In the same setting as the preceding example, with $n = 5$, the Lagrange polynomial $L_3(x)$ can be written as:

\begin{equation}
  \begin{aligned}
   \frac{L_3(x)}{\alpha_3} &= (x - \omega^0)(x - \omega^1)(x - \omega^2)(x - \omega^4) \\
        &= b_4 x^4 + b_3 x^3 + b_2 x^2 + b_1 x + b_0
  \end{aligned}
\end{equation}

The coefficients of this polynomial are:

\begin{equation}
  \begin{aligned}
    b_4 &= 1 \\
        &= (-1)^{0} \cdot S_3(0) \\
    \\
    b_3 &= -(\omega^0 + \omega^1 + \omega^2 + \omega^4) \\
        &= (-1)^{1} \cdot S_3(1) = \omega^3 \\
    \\
    b_2 &= \omega^0\omega^1 + \omega^0\omega^2 + \omega^0\omega^4 + \omega^1\omega^2 + \omega^1\omega^4 + \omega^2\omega^4 \\
        &= (-1)^{2} \cdot S_3(2) = \omega \\
    \\
    b_1 &= -(\omega^0\omega^1\omega^2 + \omega^0\omega^1\omega^4 + \omega^0\omega^2\omega^4 + \omega^1\omega^2\omega^4) \\
        &= (-1)^{3} \cdot S_3(3) = \omega^4\\
    \\
    b_0 &= \omega^0\omega^1\omega^2\omega^4 \\
        &= (-1)^{4} \cdot S_3(4) = \omega^2\\
  \end{aligned}
\end{equation}

Therefore:

\begin{equation}
  \begin{aligned}
   \frac{L_3(x)}{\alpha_3} = x^4 + \omega^3 x^3 + \omega x^2 + \omega^4 x + \omega^2
  \end{aligned}
\end{equation}

By definition, $L_3(\omega^3) = 1$. Plugging this in the above equation, we obtain:

\begin{equation}
  \begin{aligned}
   \frac{L_3(\omega^3)}{\alpha_3} = \frac{1}{\alpha_3} = 5 \omega^2 \implies \alpha_3 = \frac{1}{5 \omega^2} = \frac{\omega^3}{5}
  \end{aligned}
\end{equation}

So finally:

\begin{equation}
  \begin{aligned}
    L_3(x) = \frac{\omega^3}{5} \cdot \frac{x^5 - 1}{x - \omega^3} = \frac{1}{5} \cdot (\omega^3 x^4 + \omega x^3 + \omega^4 x^2 + \omega^2 x + 1)
  \end{aligned}
\end{equation}


% --------------------------------------------------------------------------------

\section{Explicit representations of $L(x)$ and $L_i(x)$}

By now it should be clear that we can write $L(x)$ and $L_i(x)$ as:

\begin{equation}
  \begin{aligned}
    L(x) &= \prod_{k=0}^{n-1} (x - \omega^k) = \sum_{k=0}^{n} (-1)^{n-k} S(n-k) x^k
  \end{aligned}
\end{equation}

\begin{equation}
  \begin{aligned}
    \frac{L_i(x)}{\alpha_i} &= \prod_{\substack{j=0 \\ j \neq i}}^{n-1} (x - \omega^k) = \sum_{k=0}^{n-1} (-1)^{n-1-k} S_i(n-1-k) x^k
  \end{aligned}
\end{equation}

\subsection{Values of $S(k)$ and $S_i(k)$ for $0 < k < n$}

We want to show that for $0 < k < n$, we have:

\begin{equation}
  \begin{aligned}
    S(k) &= 0 \\
    \\
    S_i(k) &= (-1)^k \omega^{ik}
  \end{aligned}
\end{equation}

We proceed by induction.

We have $S(1) = \omega^0 + \omega^1 + ... + \omega^{n-1} = 0$ because $\omega \cdot S(1) = S(1)$. We also have $S_i(1) = S(1) - \omega^i = -\omega^i$. The relation therefore holds for $k = 1$.

Let us now express $S(k)$ in $n$ different ways:

\begin{equation}
  \begin{aligned}
    S(k) &= \omega^0 S_0(k-1) + S_0(k) \\
         &= \omega^1 S_1(k-1) + S_1(k) \\
         &= ... \\
         &= \omega^{n-1} S_{n-1}(k-1) + S_{n-1}(k)
  \end{aligned}
\end{equation}

Adding all of these up, we obtain:

\begin{equation}
  \begin{aligned}
    nS(k) &= \sum_{i=0}^{n-1} \omega^iS_i(k-1) + \sum_{i=0}^{n-1} S_i(k) \\
          &= A(k) + B(k)
  \end{aligned}
\end{equation}

By the induction hypothesis, we have $S_i(k-1) = (-1)^{k-1} \omega^{i(k-1)}$ and therefore:

\begin{equation}
  \begin{aligned}
    A(k) &= \sum_{i=0}^{n-1} \omega^i (-1)^{k-1} \omega^{i(k-1)} = (-1)^{k-1} \sum_{i=0}^{n-1} \omega^{ik}\\
  \end{aligned}
\end{equation}

Notice that $\omega^k A(k) = A(k)$ and therefore $A(k) = 0$.

As to $B(k)$, it has to be a multiple of S(k). To see why, take an arbitrary term $\omega^{j_1 + .. + j_k}$ of $S(k)$. This term cannot appear in any of the $k$ sums $S_{j_1}(k)$, $S_{j_2}(k)$ ... $S_{j_k}(k)$, but will appear in all of the $n-k$ other groups. Therefore $B(k) = (n-k)S(k)$.

A second way to looks at this is by invoking symmetry. Taking some arbitrary term of $S(k)$, it will appear in $B(k)$ a certain number of times. There is no reason why one term would appear more or less often than another, meaning they all appear with the same frequency. As $nS(k)$ has $n\binom{n}{k}$ terms and $B(k)$ has $n\binom{n-1}{k}$ terms, we again conclude that $B(k) = (n-k)S(k)$. As a result: 

\begin{equation}
  \begin{aligned}
    nS(k) &= A(k) + B(k) = 0 + (n-k)S(k) \\
    &\implies kS(k) = 0 \\
    &\implies S(k) = 0
  \end{aligned}
\end{equation}

We now use this fact to determine the value of $S_i(k)$.

\begin{equation}
  \begin{aligned}
    S(k) &= \omega^i S_i(k-1) + S_i(k) = 0\\
    &\implies S_i(k) = - \omega^i S_i(k-1) \\
  \end{aligned}
\end{equation}

Again using the induction hypothesis, we obtain:

\begin{equation}
  \begin{aligned}
    S_i(k) = - \omega^i (-1)^{k-1} \omega^{i(k-1)} = (-1)^k \omega^{ik}
  \end{aligned}
\end{equation}

This completes the proof.

\subsection{An example}

Let's take the example of $S(2)$ with $n = 4$:

\begin{equation}
  \begin{aligned}
    S(2) &= \omega^0\omega^1 + \omega^0\omega^2 + \omega^0\omega^3 + \omega^1\omega^2 + \omega^1\omega^3 + \omega^2\omega^3 \\
         \\
         &= \omega^0 (\omega^1 + \omega^2 + \omega^3) +  (\omega^1\omega^2 + \omega^1\omega^3 + \omega^2\omega^3) = \omega^0 S_0(1) + S_0(2) \\
         &= \omega^1 (\omega^0 + \omega^2 + \omega^3) +  (\omega^0\omega^2 + \omega^0\omega^3 + \omega^2\omega^3) = \omega^1 S_1(1) + S_1(2) \\
         &= \omega^2 (\omega^0 + \omega^1 + \omega^3) +  (\omega^0\omega^1 + \omega^0\omega^3 + \omega^1\omega^3) = \omega^2 S_2(1) + S_2(2) \\
         &= \omega^3 (\omega^0 + \omega^1 + \omega^2) +  (\omega^0\omega^1 + \omega^0\omega^2 + \omega^1\omega^2) = \omega^3 S_3(1) + S_3(2) \\
  \end{aligned}
\end{equation}

So we can write $4S(2) = A(2) + B(2)$, with:

\begin{equation}
  \begin{aligned}
    A(2) &= \omega^0 S_0(1) + \omega^1 S_1(1) + \omega^1 S_1(1) + \omega^2 S_2(1) + \omega^3 S_3(1) \\
    B(2) &= S_0(2) + S_1(2) + S_2(2) + S_3(2)
  \end{aligned}
\end{equation}

Remembering that $0 = \omega^0 + \omega^1 + \omega^2 + \omega^3 \implies \omega^1 + \omega^2 + \omega^3 = -\omega^0$ etc:

\begin{equation}
  \begin{aligned}
    A(2) &= \omega^0(-\omega^0) + \omega^1(-\omega^1) + \omega^2(-\omega^2) + \omega^3(-\omega^3) \\
    &= -2(\omega^0 + \omega^2) \\
    &= 0
  \end{aligned}
\end{equation}

Now note that every term of $S(2)$ appears in $B(2)$ exactly twice, which means that $B(2) = 2S(2)$. 

\begin{equation}
  \begin{aligned}
    4S(2) = A(0) + B(2) = 0 + 2S(2) \implies S(2) = 0\\
  \end{aligned}
\end{equation}

Finally, we find $S_0(2) = S_2(2) = \omega^0 = 1$ and $S_1(2) = S_3(2) = \omega^2 = -1$.  


% --------------------------------------------------------------------------------

\subsection{The value of $S(n)$}

We have $S(n) = \omega^0\omega^1...\omega^{n-1}$ and therefore $S(n) = \omega^{0+1+...+(n-1)} = \omega^{\nicefrac{n(n-1)}{2}}$.

When $n$ is even, we can write $n = 2m$ and $S(n) = \omega^{m(2m-1)} = \omega^{-m} = \omega^{m} = -1$. To see why $\omega^{m} = -1$, note that $\omega^{m}(1 + \omega^{m}) = 1 + \omega^{m}$.

When $n$ is odd, we can write $n = 2m + 1$ and thus $S(0) = \omega^{nm} = 1$.

We can therefore write:

\begin{equation}
  S(n) = (-1)^{n+1}
\end{equation}


% --------------------------------------------------------------------------------

\subsection{The formula for $L(x)$}

Now that we know the formula for $S(n)$, we can apply this to $L(x)$:

\begin{equation}
  \begin{aligned}
    L(x) &= \sum_{k=0}^{n} (-1)^{n-k} S(n-k) x^k \\
         &= (-1)^n S(n) + \sum_{k=1}^{n-1} (-1)^{n-k} S(n-k) x^k + S(0) x^n \\
         &= (-1)^n (-1)^{n+1} + 0 + 1 \cdot x^n \\
         &= -1 + x^n
  \end{aligned}
\end{equation}

As a result, we have:

\begin{equation}
  \begin{aligned}
    L(x) &= \prod_{k=0}^{n-1} (x - \omega^k) = x^n - 1
  \end{aligned}
\end{equation}


% --------------------------------------------------------------------------------

\subsection{The value of $\alpha_i$}

Now that we know $S_i(k)$, we can write:

\begin{equation}
  \begin{aligned}
    \frac{L_i(x)}{\alpha_i} &= \sum_{k=0}^{n-1} (-1)^{n-1-k} S_i(n-1-k) x^k \\
                            &= \sum_{k=0}^{n-1} (-1)^{n-1-k} (-1)^{n-1-k} \omega^{i(n-1-k)} x^k \\
                            &= \sum_{k=0}^{n-1} \omega^{-i(k+1)} x^k
  \end{aligned}
\end{equation}

By definition, $L_i(\omega^i) = 1$, so we can write:

\begin{equation}
  \begin{aligned}
    \frac{L_i(\omega^i)}{\alpha_i} &= \frac{1}{\alpha_i} = \sum_{k=0}^{n-1} \omega^{-i(k+1)} \omega^{ik} = n \cdot \omega^{-i}
  \end{aligned}
\end{equation}

So we finally have $\alpha_i = \omega^i / n$.


% --------------------------------------------------------------------------------

\subsection{The formula for $L_i(x)$}

Now that we have $\alpha_i$, we can restate 

\begin{equation}
  \begin{aligned}
    L_i(\omega^i) &= \frac{\omega^i}{n} \cdot \sum_{k=0}^{n-1} \omega^{-i(k+1)} x^k = \frac{1}{n} \cdot \sum_{k=0}^{n-1} \omega^{-ik} x^k
  \end{aligned}
\end{equation}

Final formula:

\begin{equation}
  \begin{aligned}
    L_i(x) &= \frac{\omega^i}{n} \cdot \prod_{\substack{j=0 \\ j \neq i}}^{n-1} (x - \omega^k) \\
           &= \frac{\omega^i}{n} \cdot \frac{x^n - 1}{x - \omega^i} \\
           &= \frac{1}{n} \cdot \sum_{k=0}^{n-1} \omega^{-ik} x^k
  \end{aligned}
\end{equation}



%%%%%%%%%%%%%%%%%%%%%%%%%%%%%%%%%%%%%%%%%%%%%%%%%%%%%%%%%%%%%%%%%%%%%%%%%%%%%%%%
%
% Bibliography
%

\bibliography{bib/lagrange_poly.bib}
\bibliographystyle{unsrt}

\end{document}
